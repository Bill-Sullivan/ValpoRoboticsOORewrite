Use Instructions\+:

This is the multi-\/file code for the robotic football team. The design philosophy behind this project is to divide the code amoung a number of classes to make it easy to maintain and share code between robots. The abilities of the robot to be programmed are then chosen from a list of define statements in \mbox{\hyperlink{_config_8hpp_source}{config.\+hpp}}, like so\+:

In \mbox{\hyperlink{_config_8hpp_source}{config.\+hpp}}\+: 
\begin{DoxyCode}
\{c++\}
#define QUARTERBACK
//#define LINEMAN
//#define RECIEVER
//#define RUNNINGBACK
//#define CENTER
//#define NONQB\_OMNI
//#define KICKER

//#define BASIC\_DRIVETRAIN
//#define OMNIWHEEL\_DRIVETRAIN
//#define CENTER\_PERIPHERALS
//#define QB\_PERIPHERALS
//#define KICKER\_PERIPHERALS
//#define RECEIVER\_PERIPHERALS
//#define LED\_STRIP
//#define TACKLE

//#define CIM\_MOTOR
//#define \_775\_MOTOR
//#define BANEBOTS\_MOTOR

#define BAG\_MOTOR
\end{DoxyCode}
 This example would create a robot that uses a bag motor omniwheel drivetrain (specifically our quarterback), with the led strip, quarterback arm, tackle sensor enabled. In \mbox{\hyperlink{_error_handeling_8hpp_source}{Error\+Handeling.\+hpp}} there are a list of error cases that the compiler goes through to make sure no incompatable options are chosen.

Aternatively you can manually pick what options you want to enable. This can be useful when developing a new robot or diagnosing problems with individual parts on an existing robot.

In \mbox{\hyperlink{_config_8hpp_source}{config.\+hpp}}\+: 
\begin{DoxyCode}
\{c++\}
//#define QUARTERBACK
//#define LINEMAN
//#define RECIEVER
//#define RUNNINGBACK
//#define CENTER
//#define NONQB\_OMNI
//#define KICKER

#define PERF\_BOARD\_SHIELD
//#define PCB\_SHIELD

//#define BASIC\_DRIVETRAIN    //uncomment for 2 drive wheels
//#define DUAL\_MOTORS
//#define LR\_TACKLE\_PERIPHERALS         //uncomment for special handicap for the tackles
//#define OMNIWHEEL\_DRIVETRAIN  //uncomment for omniwheel robots
#define NEW\_OMNIWHEEL\_DRIVETRAIN

//#define CENTER\_PERIPHERALS  //uncomment for special features of center 
#define QB\_PERIPHERALS      //uncomment for special QB features
//#define IR\_MAST
//#define QB\_TRACKING
//#define KICKER\_PERIPHERALS  //uncomment for special Kicker features
//#define RECEIVER\_PERIPHERALS  
#define LED\_STRIP       //uncomment for LED functionality
#define TACKLE          //uncomment for tackle sensor functionality
//#define ROTATION\_LOCK

//#define CIM\_MOTOR
//#define \_775\_MOTOR
//#define BANEBOTS\_MOTOR

#define BAG\_MOTOR
\end{DoxyCode}


This example does the same thing as the above example.

\subsection*{\#\#\#\# Useful Libraries }


\begin{DoxyEnumerate}
\item (P\+S3 Integration) \href{https://github.com/felis/USB_Host_Shield_2.0}{\tt U\+SB Host\+Shield 2.\+0}
\item (Omniwheel Rotation Locking) \href{https://github.com/adafruit/Adafruit_BNO055}{\tt Adafruit B\+N\+O055}
\end{DoxyEnumerate}

\subsection*{\#\#\#\# Controls }


\begin{DoxyItemize}
\item {\bfseries Basic Drivetrain}
\begin{DoxyItemize}
\item {\itshape Up/\+Down Left Joystick} -\/ Forward and Backward movement
\item {\itshape Left/\+Right Right Joystick} -\/ Turning
\item {\itshape R2} -\/ activates \char`\"{}boost\char`\"{}
\item {\itshape Start} -\/ Puts robot in \char`\"{}kids mode\char`\"{}. The speed is reduced, boost is disabled, and the leds will change
\item {\itshape Select \+\_\+-\/ Calibration mode -\/ disables drivetrain while changes are made
\begin{DoxyItemize}
\item \+\_\+\+Up/\+Down D-\/\+Pad -\/ compensates for drag left or right
\item {\itshape Select} -\/ exit Calibration Mode to regular drive mode
\end{DoxyItemize}}
\end{DoxyItemize}
\item {\itshape {\bfseries \mbox{\hyperlink{class_center}{Center}}}
\begin{DoxyItemize}
\item \mbox{\hyperlink{class_center}{Center}} currently uses basic drivetrain
\item {\itshape T\+R\+I\+A\+N\+G\+LE} -\/ raise the center release servo
\item {\itshape C\+R\+O\+SS} -\/ lower the center release servo
\end{DoxyItemize}}
\item {\itshape {\bfseries Omniwheel Drivetrain}
\begin{DoxyItemize}
\item {\itshape Up/\+Down/\+Left/\+Right Left Joystick} -\/ Lateral movement in any direction
\item {\itshape Up/\+Down/\+Left/\+Right D-\/\+Pad}-\/ Lateral Movement along compass directions at full power
\item {\itshape Left/\+Right Right Joystick} -\/ Turning -\/ as of version 1.\+0.\+3 this will disable rotation correction
\item {\itshape R3} (Right Joystick Press) -\/ Re-\/engage rotation correction
\item {\itshape R2} -\/ slow down speed
\item {\itshape L1} -\/ reverse directions (make back of robot front and vise versa)
\item Throwing
\begin{DoxyItemize}
\item {\itshape S\+Q\+U\+A\+RE} -\/ Handoff throw
\item {\itshape C\+R\+O\+SS} -\/ Reciever handoff throw/weak toss
\item {\itshape C\+I\+R\+C\+LE} -\/ mid range throw
\item {\itshape T\+R\+I\+A\+N\+G\+LE} -\/ max power throw
\item {\itshape R1} -\/ return thrower to down position
\item {\itshape L2} -\/ hold to enable throw offset
\begin{DoxyItemize}
\item {\itshape Up/\+Down D-\/\+Pad} -\/ adjust power of all throws but triangle
\end{DoxyItemize}
\end{DoxyItemize}
\end{DoxyItemize}}
\item {\itshape {\bfseries \mbox{\hyperlink{class_kicker}{Kicker}}}
\begin{DoxyItemize}
\item \mbox{\hyperlink{class_kicker}{Kicker}} -\/ currently uses basic drivetrain
\item {\itshape C\+R\+O\+SS} -\/ kick
\item {\itshape T\+R\+I\+A\+N\+G\+LE} -\/ reload
\end{DoxyItemize}}

{\itshape  Controls for the current M\+R\+DC Robots\+: \mbox{\hyperlink{md__ernie_controls}{Ernie\textquotesingle{}s Controls}}}

{\itshape  Marvin\+Controls.\+md}

{\itshape  Parker\+Controls.\+md ~\newline
 

}
\end{DoxyItemize}

{\itshape To create a new peripheral}

{\itshape First copy the example peripheral folder to where you want to place your peripheral usually somewhere in M\+R\+D\+C\+Peripherals or Robotic\+Football\+Peripherals}

{\itshape Second implement the fallowing methods each peripheral\textquotesingle{}s instance of each method will run sequentally with the robot\textquotesingle{}s other peripherals if any}

{\itshape The Setup method should contain code to run once when the microcontroller turns on}

{\itshape The do\+Thing method should contain code to run continuously when the controller is connected}

{\itshape The do\+Not\+Connected\+Thing method should contain code to run continuously when the controller is disconnected and should stop any motors or other devices that could hurt a human opperator}

{\itshape Third include your peripherals .hpp file in either \mbox{\hyperlink{_robotic_football_libraries_8hpp_source}{Robotic\+Football\+Libraries.\+hpp}} or the appropreate M\+R\+DC robot Robot\+Name\+Libraries.\+hpp file i.\+e\+: \mbox{\hyperlink{_ernie_libraries_8hpp_source}{Ernie\+Libraries.\+hpp}}}

{\itshape Fourth in \mbox{\hyperlink{_robot_8hpp_source}{Robot.\+hpp}} add your \mbox{\hyperlink{class_peripheral}{Peripheral}} to peripheral\+Vec}

{\itshape To do this look for a series of statements in the setup method that look roughtly like this\+:}

{\itshape In \mbox{\hyperlink{_config_8hpp_source}{config.\+hpp}}\+: 
\begin{DoxyCode}
\{c++\}
#if defined(YOUR\_PERIPHERAL) 
  peripheralVec.push\_back(new YourPeripheral);
#endif
\end{DoxyCode}
}

{\itshape and replace Y\+O\+U\+R\+\_\+\+P\+E\+R\+I\+P\+H\+E\+R\+AL with your peripherals name in a similar format}

{\itshape and replace Your\+Peripheral with your peripherals name in a similar format}

{\itshape The code you wrote in the your peripheral should now run }